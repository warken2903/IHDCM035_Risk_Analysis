%----------------------------------------------------------------------------------------
%	PACKAGES AND OTHER DOCUMENT CONFIGURATIONS
%----------------------------------------------------------------------------------------
\documentclass[12pt]{article}
\input{structure.tex}
\bibliography{biblio}
\begin{document}
%----------------------------------------------------------------------------------------
%	TITLE PAGE
%----------------------------------------------------------------------------------------
\begin{titlepage}
	\newcommand{\HRule}{\rule{\linewidth}{0.5mm}}
	\center
	%------------------------------------------------
	%	Headings
	%------------------------------------------------
	\textsc{\LARGE Université de Namur}\\[1.5cm]
	\textsc{\Large Sécurité et Fiabilité des Systèmes Informatiques }\\[0.5cm]
	\textsc{\large IHDCM035}\\[0.5cm]
	%------------------------------------------------
	%	Title
	%------------------------------------------------	
	\HRule\\[0.4cm]
	{\huge\bfseries Etude des Risques: Informatisation d'un Centre Hospitalier}\\[0.4cm]
	\HRule\\[1.5cm]	
	%------------------------------------------------
	%	Author(s)
	%------------------------------------------------	
	\begin{minipage}{0.4\textwidth}
		\begin{flushleft}
			\large
			\textit{Auteur}\\
			Kenny \textsc{Warszawski} 
		\end{flushleft}
	\end{minipage}
	~
	\begin{minipage}{0.4\textwidth}
		\begin{flushright}
			\large
			\textit{Professeur}\\
			Jean-Nöel \textsc{Colin} 
		\end{flushright}
	\end{minipage}	
	%------------------------------------------------
	%	Date
	%------------------------------------------------	
	\vfill\vfill\vfill
	{\large\today}
	%------------------------------------------------
	%	Logo
	%------------------------------------------------
	\vfill\vfill
	\includegraphics[width=0.2\textwidth]{assets/placeholder.png}\\[1cm]
	\vfill
\end{titlepage}

\newpage
%----------------------------------------------------------------------------------------
%	Content
%----------------------------------------------------------------------------------------
\renewcommand{\contentsname}{Table des matières}
\tableofcontents
\newpage
%----------------------------------------------------------------------------------------
%	Introduction
%----------------------------------------------------------------------------------------

\section{Introduction} 

\subsection{Contexte}

\justify
Cette étude des risques concerne le Centre Hospitalier Mercy West(CHMW). Ce centre a mis en place un système informatique qui permet de centraliser les données de leurs patients. Afin de réaliser cela, l'hopital a mis à disposition un ordinateur connecté à une plateforme en ligne. Ainsi, le corps médical peut encoder les informations nécessaires sur leurs patients à la fin de leur service. Avant de commencer leur journée, le personnel peut également accéder aux dernières informations récoltées par leurs collègues pour rester à jour sur: l'état de santé des patients, les soins reçus, les opérations subies, les médicaments prescris, etc.

\justify
Chaque membre du personnel possède un badge afin de s'authentifier sur la plateforme. Les droits de lecture et modification d'un dossier médical sont associés à des droits qui sont assignés aux utilisateurs.  Ces droits sont associés à la fonction professionelle que l'utilisateur authentifié exerce. Par exemple, si un médecin s'authentifie, il pourra modifier les prescriptions de médicaments d'un patient tandis qu'une aide soignante ne pourra pas. Par contre, cette dernière aura le droit de modifier l'état de santé général du patient: taille, poids, nutrition, etc.

\justify
Ce logiciel impacte donc le quotidien des employés de cet hopital. Il est indispensable que tout le personnel indique rigoureusement les information concernant le patient. Ainsi, il sera possible de garantir un suivi médical journalier de haute qualité mais également d'en conserver un historique. Via cette plateforme, il est également possible de gérer les stocks de médicaments. L'accès aux informations médicales, l'encodage des données ainsi que la gestion des stocks pharmaceutiques sont donc les \textbf{biens essentiels} liés à ce projet. 

\justify
La confidentialité est un des critères de sécurité les plus important pour l'hôpital. De fait, si les informations médicales d'un patient arrivent entre de mauvaises mains, cela peut également avoir des conséquences dramatiques. Il est essentiel que les données médicales soient sécurisées et exploitable uniquement par les utilisateurs qui en ont le droit.

\justify
En ce qui concerne les \textbf{biens supports}, le centre hospitalier possède une infrastructure informatique dédiée afin de faire fonctionner l'ensemble de ses logiciels. Cette infrastructure comprend: des ordinateurs, des serveurs, un sous-réseau, un système d'authentification et de multiples disques durs afin de pouvoir stocker les données.

\subsection{Objectifs}

\justify
L'objectif de cette étude est de pouvoir établir une analyse de risque concernant ce projet. De plus, un plan d'action sera proposé en réponse aux scénarios de menace et aux évènements redoutés par le centre hospitalier. Le champs de cette étude sera toutefois limitée uniquement à la plateforme en ligne précédemment mentionnée. Tous les autres processus organisationnel ou informatiques nullement liés à ce projet ne seront pas pris en compte.

\section{Analyse de risques}

\subsection{Evènement redoutés}

Cette section est dédiée à une analyse des évènements redoutés. Cette analyse est basée sur les biens essentiels de l'hôpital et des critères de sécurités importants. (Disponibilité, Confidentialité et Intégrité)

\subsubsection{Accès aux informations médicales}

L'analyse de ce bien essentiel concerne la consultation des informations des patients. Par exemple, en début de service par un membre du corps médical.

\begin{longtable}{|l|l|l|l|l|}
\hline

\multicolumn{1}{|c|}{\textbf{\begin{tabular}[c]{@{}c@{}}Evènements\\ Redoutés\end{tabular}}} & \multicolumn{1}{c|}{\textbf{\begin{tabular}[c]{@{}c@{}}Critère de\\ Sécurité\end{tabular}}} & \multicolumn{1}{c|}{\textbf{\begin{tabular}[c]{@{}c@{}}Source de\\ la Menace\end{tabular}}} & \multicolumn{1}{c|}{\textbf{Impact}} & \multicolumn{1}{c|}{\textbf{Sévérité}} \\ \hline
\endfirsthead
%
\endhead
%

	\begin{tabular}[c]{@{}l@{}}Incendie dans\\ la salle des\\ serveurs (+\\ système de\\ stockage)\end{tabular}     & Disponibilité       & \begin{tabular}[c]{@{}l@{}}- Dysfonction-\\nement de\\ matériel\\- Surtension\\ électrique\\ - Personne\\ mal-intentionnée\end{tabular}                                                                                                & \begin{tabular}[c]{@{}l@{}}- Perte des\\ données\\ - Système\\ inaccessible\\ - Réputation de\\ l'hôpital\\ - Vie des\\ patients\end{tabular} & Elevé    \\ \hline

	\begin{tabular}[c]{@{}l@{}}Dysfonction-\\nement du\\ système d'au-\\thentification\end{tabular} & Disponibilité       & \begin{tabular}[c]{@{}l@{}}- Erreur logiciel\\ - Personne\\mal-intentionnée\\- TODO\end{tabular}                                                                                                                              & \begin{tabular}[c]{@{}l@{}}- Impossibilité\\ de consulter les\\ informations\end{tabular}                                               & Elevé    \\ \hline

	\begin{tabular}[c]{@{}l@{}}Intrusion d'une\\ personne non\\-autorisée\end{tabular}          & Confidentialité     & \begin{tabular}[c]{@{}l@{}}- Personnel qui a\\ oublié son badge\\ et utilise celui\\ d'un collègue\\ - Mauvaise gestion\\ des roles as-\\signés aux utili-\\sateurs\\ - Personne\\ mal-intentionnée\end{tabular} & \begin{tabular}[c]{@{}l@{}}- Divulgation de\\ données person-\\nelles à une\\ personne non-\\autorisée.\\(secret médical)\end{tabular}      & Elevé    \\ \hline

	\begin{tabular}[c]{@{}l@{}}Panne de courant\end{tabular}          & Disponibilité     & \begin{tabular}[c]{@{}l@{}}- Problème sur\\ le réseau  éle-\\ctrique\\- Personne\\ mal-intentionnée \end{tabular} & \begin{tabular}[c]{@{}l@{}}- Impossible de\\ récupérer les\\ informations \\des patients \end{tabular}      & Elevé    \\ \hline

\caption{Table d'analyse de l'accès aux information médicales}
\label{table:accesInformationsMedicales}
\end{longtable}

\subsubsection{Encodage des données}

L'analyse de ce bien essentiel concerne l'encodage des données sur un patient. Par exemple, en fin de service par un membre du corps médical. Cependant, l'encodage requiert un formalisme précis. Les nouveaux médecins ou tout médecin non-initié à ce formalisme peut engendrer un encodage erroné. 

\begin{longtable}{|l|l|l|l|l|}
\hline
\multicolumn{1}{|c|}{\textbf{\begin{tabular}[c]{@{}c@{}}Evènements\\ Redoutés\end{tabular}}} & \multicolumn{1}{c|}{\textbf{\begin{tabular}[c]{@{}c@{}}Critère de\\ Sécurité\end{tabular}}} & \multicolumn{1}{c|}{\textbf{\begin{tabular}[c]{@{}c@{}}Source de\\ la Menace\end{tabular}}} & \multicolumn{1}{c|}{\textbf{Impact}} & \multicolumn{1}{c|}{\textbf{Sévérité}} \\ \hline
\endfirsthead
%
\endhead
%
\begin{tabular}[c]{@{}l@{}}Intrusion d'une\\ personne non\\-autorisée\end{tabular} & Intégrité & \begin{tabular}[c]{@{}l@{}}- Personnel qui a\\ oublié son badge\\ et utilise celui\\ d'un collègue\\ - Mauvaise gestion des\\ roles assignés aux\\ utilisateurs\\ - Personne\\ mal-intentionnée\end{tabular} & \begin{tabular}[c]{@{}l@{}}- Altération des\\ données des\\ patients\\ - La vie du\\ patient est\\ mise en\\ danger\end{tabular} & Elevé \\ \hline

\begin{tabular}[c]{@{}l@{}}Dysfonction-\\nement du\\ système d'au-\\thentification\end{tabular} & Disponibilité & \begin{tabular}[c]{@{}l@{}}- Erreur logiciel\\ - Utilisateur\end{tabular} & \begin{tabular}[c]{@{}l@{}}- Système\\ inutilisable \end{tabular} & Elevé \\ \hline

\begin{tabular}[c]{@{}l@{}}Panne de\\ courant\end{tabular} & Disponibilité & \begin{tabular}[c]{@{}l@{}}- Problème sur\\ le réseau électrique\\ - Personne\\ mal-intentionnée\end{tabular} & \begin{tabular}[c]{@{}l@{}}- Système\\ inutilisable \end{tabular} & Elevé \\ \hline

\begin{tabular}[c]{@{}l@{}}Incendie dans\\ la salle des\\ serveurs (+\\ système de\\ stockage)\end{tabular} & Disponibilité & \begin{tabular}[c]{@{}l@{}}Dysfonction-\\nement du\\ matériel\\- Surtension\\ électrique\\ - Personne\\ mal-intentionnée\end{tabular} & \begin{tabular}[c]{@{}l@{}}- Système\\ inutilisable \end{tabular} & Elevé \\ \hline

\caption{Table d'analyse de l'encodage des données}
\label{table:encodageDonnees}
\end{longtable}

\subsubsection{Gestion des stocks pharmaceutiques}

Ce bien essentiel correspond à la partie du système informatique qui est capable de gérer les stocks pharmaceutique. Etant donné que les médicaments sont prescris aux patients de manière informatisée, il est possible pour le gestionnaire de stocks d'accéder à une estimation des médicaments qui restent en stocks et également les médicaments qu'il faudrait commander dans les prochains jours. Grace à ce système, il peut optimiser au mieux les stocks afin de ne pas tomber en rupture de médicaments. A cette fin, le système prévoit également la possibilité de programmer des réservations aux fournisseurs de manière automatisée. 

\begin{longtable}{|l|c|l|l|c|}
\hline
\multicolumn{1}{|c|}{\textbf{\begin{tabular}[c]{@{}c@{}}Evènements\\ Redoutés\end{tabular}}} & \multicolumn{1}{c|}{\textbf{\begin{tabular}[c]{@{}c@{}}Critère de\\ Sécurité\end{tabular}}} & \multicolumn{1}{c|}{\textbf{\begin{tabular}[c]{@{}c@{}}Source de\\ la Menace\end{tabular}}} & \multicolumn{1}{c|}{\textbf{Impact}} & \multicolumn{1}{c|}{\textbf{Sévérité}} \\ \hline
\endfirsthead
%
\endhead
%
\begin{tabular}[c]{@{}l@{}}Incendie dans\\ la salle des\\ serveurs\end{tabular} & Disponibilité & \begin{tabular}[c]{@{}l@{}}- Dysfonction-\\ nement du\\ matériel\\ - Surtension\\ électrique\\ - Personne\\ mal-intentionnée\end{tabular} & \begin{tabular}[c]{@{}l@{}}- Impossible de\\ consulter le \\ stock restant\\ - Impossible\\ de réapprovi-\\ sionner les\\ stocks\end{tabular} & Elevé \\ \hline

\begin{tabular}[c]{@{}l@{}}Dosage de mé-\\ dicaments er-\\ ronés\end{tabular} & Intégrité & \begin{tabular}[c]{@{}l@{}}- Nouveaux mé-\\ decins ne maitri-\\ sant pas le logi-\\ ciel\\ - Personne mal-\\ intentionnée\end{tabular} & \begin{tabular}[c]{@{}l@{}}- Trop de com-\\ mandes => \\ Mise à mal de\\ la finance de l'\\ hôpital\\ - Trop peu de\\ commande =>\\ Pas assez de \\ médicaments\\ pour les patients\end{tabular} & Elevé \\ \hline

\begin{tabular}[c]{@{}l@{}}Mauvaise pro-\\ grammation du\\ système auto-\\ matisé de ges-\\ tion des stocks\end{tabular} & Intégrité & \begin{tabular}[c]{@{}l@{}}- Nouveaux mé-\\ decins ne maitri-\\ sant pas le logi-\\ ciel\\ - Personne mal-\\ intentionnée\end{tabular} & \begin{tabular}[c]{@{}l@{}}- Dosage de mé\\ dicaments er-\\ ronées\end{tabular} & Elevé \\ \hline

\begin{tabular}[c]{@{}l@{}}Intrusion d'une\\ personne non-\\ authorisée\end{tabular} & Intégrité & \begin{tabular}[c]{@{}l@{}}- Personnel qui a\\ oublié son badge\\ et utilise celui\\ d'un collègue\\ - Mauvaise gestion\\ des roles as-\\ signés aux utili-\\ sateurs\\ - Personne\\ mal-intentionnée\end{tabular} & \begin{tabular}[c]{@{}l@{}}- Sabotage des \\ stocks informa-\\ tisés\end{tabular} & Elevé \\ \hline

\begin{tabular}[c]{@{}l@{}}Panne de cou-\\ rant\end{tabular} & Disponibilité & \begin{tabular}[c]{@{}l@{}}- Problème sur\\ le réseau électri-\\ que\\ \\ - Personne mal-\\ intentionnée\end{tabular} & \begin{tabular}[c]{@{}l@{}}- Impossible de\\ consulter le \\ stock restant\\ - Impossible\\ de réapprovi-\\ sionner les\\ stocks\end{tabular} & Elevé \\ \hline

\begin{tabular}[c]{@{}l@{}}Dysfonction-\\ nement du sys-\\ tème d'authenti-\\ fication\end{tabular} & Disponibilité & \begin{tabular}[c]{@{}l@{}}- Erreur Logiciel\\ - Personne\\ mal-intentionnée\end{tabular} & \begin{tabular}[c]{@{}l@{}}- Impossible de\\ consulter le \\ stock restant\\ - Impossible\\ de réapprovi-\\ sionner les\\ stocks\end{tabular} & Elevé \\ \hline
\caption{Table d'analyse de gestion des stocks pharmaceutiques}
\label{tab:gestion-stock-pharmacie}\\
\end{longtable}

\subsection{Scénarios de menace}

Cette section est dédiée à une analyse des scénarios de menaces. Cette analyse est basée sur les biens support de l'hôpital.

\subsubsection{Serveurs}

Les serveurs de l'hôpital sont situés dans une salle qui est prévue à cet effet. Historiquement, l'hôpital ne possédait que très peu de logiciels informatiques. Ils n'ont donc pas investi dans des équipements afin de protéger leur infrastructure. Cette catégorie reprend donc les serveurs où sont installés les logiciels de l'hôpital.

\begin{table}[ht]
\begin{tabular}{|l|l|l|}
\hline
Scénario de menace & Source de la menace & Probabilité \\ \hline
Panne du serveur & \begin{tabular}[c]{@{}l@{}}Obsolescence du matériel\end{tabular} & Moyenne \\ \hline
Incendie & \begin{tabular}[c]{@{}l@{}}- Problème matériel\\- Personne mal-intentionnée\end{tabular} & Moyenne \\ \hline
 &  &  \\ \hline
\end{tabular}
\end{table}

\subsubsection{Système d'authentification}
Le système d'authentification de l'hôpital est situé dans la même salle où se trouvent les serveurs. Ce système se trouve sur une machine dédiée et fait appel à un Active Directory afin de pouvoir authentifier les utilisateurs sur les différentes applications.

\begin{table}[ht]
\begin{tabular}{|l|l|l|}
\hline
Scénario de menace & Source de la menace & Probabilité \\ \hline
Panne de serveur & \begin{tabular}[c]{@{}l@{}}Obsolescence du matériel\end{tabular} & Moyenne \\ \hline
Incendie & \begin{tabular}[c]{@{}l@{}}- Problème matériel\\- Personne mal-intentionnée\end{tabular} & Moyenne \\ \hline
 &  &  \\ \hline
\end{tabular}
\end{table}

\subsubsection{Données}

\subsubsection{Réseau interne}

\subsubsection{Disques durs}

\subsubsection{Ordinateurs}

\subsubsection{Active Directory}

\subsection{Synthèse}

\section{Plan d'action}

\subsection{sub-1}

\subsection{sub-2}

\newpage

%---------------------------------------List of Figures----------------------------------------

\section*{Table des figures}
\makeatletter
\@starttoc{lof}% Print List of Figures
\makeatother
\addcontentsline{toc}{section}{\protect\numberline{}Table des figures}%

%---------------------------------------References----------------------------------------

\nocite{*}
\section*{Bibliographie}
\printbibliography[heading=none]
\addcontentsline{toc}{section}{\protect\numberline{}Bibliographie}%

%----------------------------------------------------------------------------------------
%----------------------------------------THE END-----------------------------------------
%----------------------------------------------------------------------------------------

\end{document}
