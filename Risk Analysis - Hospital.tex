%----------------------------------------------------------------------------------------
%	PACKAGES AND OTHER DOCUMENT CONFIGURATIONS
%----------------------------------------------------------------------------------------
\documentclass[12pt]{article}
\input{structure.tex}
\bibliography{biblio}
\begin{document}
%----------------------------------------------------------------------------------------
%	TITLE PAGE
%----------------------------------------------------------------------------------------
\begin{titlepage}
	\newcommand{\HRule}{\rule{\linewidth}{0.5mm}}
	\center
	%------------------------------------------------
	%	Headings
	%------------------------------------------------
	\textsc{\LARGE Université de Namur}\\[1.5cm]
	\textsc{\Large Sécurité et Fiabilité des Systèmes Informatiques }\\[0.5cm]
	\textsc{\large IHDCM035}\\[0.5cm]
	%------------------------------------------------
	%	Title
	%------------------------------------------------	
	\HRule\\[0.4cm]
	{\huge\bfseries Etude des Risques: Informatisation d'un Centre Hospitalier}\\[0.4cm]
	\HRule\\[1.5cm]	
	%------------------------------------------------
	%	Author(s)
	%------------------------------------------------	
	\begin{minipage}{0.4\textwidth}
		\begin{flushleft}
			\large
			\textit{Auteur}\\
			Kenny \textsc{Warszawski} 
		\end{flushleft}
	\end{minipage}
	~
	\begin{minipage}{0.4\textwidth}
		\begin{flushright}
			\large
			\textit{Professeur}\\
			Jean-Nöel \textsc{Colin} 
		\end{flushright}
	\end{minipage}	
	%------------------------------------------------
	%	Date
	%------------------------------------------------	
	\vfill\vfill\vfill
	{\large\today}
	%------------------------------------------------
	%	Logo
	%------------------------------------------------
	\vfill\vfill
	\includegraphics[width=0.2\textwidth]{assets/placeholder.png}\\[1cm]
	\vfill
\end{titlepage}

\newpage
%----------------------------------------------------------------------------------------
%	Content
%----------------------------------------------------------------------------------------
\renewcommand{\contentsname}{Table des matières}
\tableofcontents
\newpage
%----------------------------------------------------------------------------------------
%	Introduction
%----------------------------------------------------------------------------------------

\section{Introduction} 

\subsection{Contexte}

\justify
Cette étude des risques concerne le Centre Hospitalier Mercy West(CHMW). Ce centre a mis en place un système informatique qui permet de centraliser les données de leurs patients. Afin de réaliser cela, l'hopital a mis à disposition un ordinateur connecté à une plateforme en ligne. Ainsi, le corps médical peut encoder les informations nécessaires sur leurs patients à la fin de leur service. Avant de commencer leur journée, le personnel peut également accéder aux dernières informations récoltées par leurs collègues pour rester à jour sur: l'état de santé des patients, les soins reçus, les opérations subies, les médicaments prescris, etc.

\justify
Chaque membre du personnel possède un nom d'utilisateur ainsi qu'un mot de passe pour s'authentifier sur la plateforme. Les droits de lecture et modification d'un dossier médicale sont associés à des droits qui seront assignés aux utilisateurs.  Ces droits sont associés à la fonction professionelle que l'utilisateur authentifié exerce. Par exemple, si un médecin s'authentifie, il pourra modifier les prescriptions de médicaments d'un patient tandis qu'une aide soignante ne pourra pas. Par contre, cette dernière aura le droit de modifier l'état de santé général du patient: poids, TODO, TODO, TODO. (demander à mam)

\justify
Ce logiciel impacte donc le quotidien des employés de cet hopital. Il est indispensable que tout le personnel indique rigoureusement les information concernant le patient. Ainsi, il sera possible de garantir un suivi médical journalier de haute qualité mais également d'en conserver un historique. Si une personne mal intentionnée ou tout simplement non-formée indique des informations erronées, ces peuvent avoir de lourdes conséquences sur le patient. Par conséquent, la réputation de l'hopital pourrait gravement diminuer. La réputation, la qualité des soins et le suivi médical sont des assets essentiels pour cet hopital. 

\justify
La confidentialité est également un élément très important. De fait, si les informations médicales d'un patient arrivent entre de mauvaises mains, cela peut également avoir des conséquences dramatiques. Il est essentiel que les données médicales soient sécurisées et exploitable uniquement par les utilisateurs qui en ont le droit.

\justify
Le centre hospitalier possède une infrastructure informatique afin de faire fonctionner l'ensemble de ses logiciels. Les données sont stockées sur de multiples disques durs configurés en RAID 50. Ce qu'il va leur permettre de récupérer leurs données en cas de panne d'un disque par grappe. Il s'agit ici d'un énorme risque. Si l'infrastructure informatique de l'hopital est incendiée, toutes les informations des patients sont perdues.

\subsection{Objectifs}

\justify
L'objectif de cette étude est de pouvoir établir une analyse de risque concernant ce projet. De plus, un plan d'action sera proposé en réponse aux évènements redoutés par le centre hospitalier. Le champs de cette étude sera toutefois limitée uniquement à la plateforme précédemment mentionnée. Tous les autres processus organisationnel ou informatiques nullement liés à la problématique ne seront pas mentionnés.

\newpage

\section{Analyse de risques}

\subsection{sub-1}

\subsection{sub-2}

\section{Plan d'action}

\subsection{sub-1}

\subsection{sub-2}


\newpage

%---------------------------------------List of Figures----------------------------------------

\section*{Table des figures}
\makeatletter
\@starttoc{lof}% Print List of Figures
\makeatother
\addcontentsline{toc}{section}{\protect\numberline{}Table des figures}%

%---------------------------------------References----------------------------------------

\nocite{*}
\section*{Bibliographie}
\printbibliography[heading=none]
\addcontentsline{toc}{section}{\protect\numberline{}Bibliographie}%

%----------------------------------------------------------------------------------------
%----------------------------------------THE END-----------------------------------------
%----------------------------------------------------------------------------------------

\end{document}
